%-------------------------- begin preamble --------------------------
\documentclass[twoside]{article}



\usepackage{relsize,epsfig,makeidx,amsmath,amsfonts}
\usepackage[latin1]{inputenc}
%%\usepackage{minted} % packages needed for verbatim environments
\begin{document}
\section{Introduction}
\label{Introduction}
In this text we will addresses the two-dimensional, standard, linear wave equation, with damping and reflecting boundaries. We will develop a scheme for solving it and apply it to the problem of a tsunami over an uneven seabed. Especial focus will be put on the question of which kinds of seabeds causes numerical instability. 
\section{Mathematical model}
\label{Model}
For x,y in the domain $D = [0,L_y] \times [0,L_x]$ and $t \in [0,T]$ we will study the following set  of equations:
\begin{align}
u_tt + b u_t &= (q(x,y)u_x)_x + (q(x,y)u_y)_y +f(x,y,t), &x,y \in D, t \in [0, T] \\
\frac{\delta u}{\delta n} &= 0 &x,y \in \delta D,  t \in [0, T] \\
u(x,y,0) &= I(x,y) &x,y \in \delta D \\
u_t(x,y,0) &= V(x,y) &x,y \in \delta D
\end{align}

\section{Numerical scheme}
\label{Scheme}
\subsection{Numerical scheme for the inner points}
In operator notation we want to use the following scheme for inner points:
\begin{align}
[D_t D_t u + b D_2t u = D_x(q D_x u) + D_y (q D_y u) +f]_{i\, j}^n \label{PDE}
\end{align}

Calculating each summand we find:

\begin{align}
[D_t D_t u]_{i\, j}^n &= \frac{u^{n+1}_{i,j} - 2u^{n}_{i,j} + u^{n-1}_{i,j}}{\Delta t^2} \label{Dttu}\\
[bD_2t u]_{i\, j}^n &= b\frac{u^{n+1}_{i,j} - u^{n-1}_{i,j}}{2\Delta t} \label{bD2tu}\\
[D_x q D_x u]_{i\, j}^n &= \frac{q_{i +\frac{1}{2},j}(u^{n}_{i+1,j} - u^{n}_{i,j}) - q_{i -\frac{1}{2},j}(u^{n}_{i,j} - u^{n}_{i-1,j})}{\Delta x^2} \label{DxqDxu}\\
[D_y q D_y u]_{i\, j}^n &= \frac{q_{i,j +\frac{1}{2}}(u^{n}_{i,j+1} - u^{n}_{i,j}) - q_{i ,j-\frac{1}{2}}(u^{n}_{i,j} - u^{n}_{i,j-1})}{\Delta x^2} \label{DyqDyu}\\
[f]_{i\, j}^n &= f(x_i, y_i, t_i) \label{f}
\end{align}
Inserting these expressions in (\ref{PDE}) and solving for $u^{n+1}_{i,j}$ we get the following scheme for inner points:
\begin{align*}
u^{n+1}_{i,j} &= \frac{2u^{n}_{i,j} + (\frac{1}{2}b\Delta t-1.0) u^{n-1}_{i,j}}{1 + \frac{1}{2}b\Delta t} \\
&+ \frac{\Delta t^2}{\Delta x^2(\frac{1}{2}b\Delta t-1.0)}({q_{i +\frac{1}{2},j}(u^{n}_{i+1,j} - u^{n}_{i,j}) - q_{i -\frac{1}{2},j}(u^{n}_{i,j} - u^{n}_{i-1,j})}) \\ 
&+  (\frac{\Delta t^2}{\Delta y^2(\frac{1}{2}b\Delta t-1.0)} (q_{i,j +\frac{1}{2}}(u^{n}_{i,j+1} - u^{n}_{i,j}) - q_{i ,j-\frac{1}{2}}(u^{n}_{i,j} - u^{n}_{i,j-1}))\\
 &+ \frac{\Delta t^2} {\frac{1}{2}b\Delta t-1.0} f(x_i, y_i, t_i)
\end{align*}

\subsection{Numerical scheme for the first time step}
For the first time step $u^{n-1}_{i,j}$ isn't a part of the grid. To circumvent this problem we have to apply the intial condition $u_t(x,y,0) = V(x,y) \text{ for } x,y \in \delta D$. We discretize it and get:
$$[D_{2t} u = V]_{i\,j}^0 \implies \frac{u_{i\,j}^1 - u_{i\,j}^{-1}}{2\Delta t} = V(x_i,y_j) \implies u_{i\,j}^{-1} = u_{i\,j}^{1} - 2\Delta t V(x_i,y_j)$$
Inserting this in equation (\ref{Dttu}) and (\ref{bD2tu}) we get:
\begin{align}
[D_t D_t u]_{i\, j}^n &= 2(\frac{u^{n+1}_{i,j} - u^{n}_{i,j}}{\Delta t^2}  - \frac{V(x_i,y_i)}{\Delta t}) \\
[bD_2t u]_{i\, j}^n &= b\frac{V(x_i,y_i)}{\Delta t} \label{bD2tu}
\end{align}
Inserting these expressions in (\ref{PDE}) and solving for $u^{n+1}_{i,j}$ we get the following scheme for inner points in the first time step:
\begin{align*}
u^{n+1}_{i,j} &= u^{n}_{i,j} + (1 - \frac{1}{2}b\Delta t)*\Delta t V^{n-1}_{i,j} \\
&+ \frac{\Delta t^2}{2\Delta x^2}({q_{i +\frac{1}{2},j}(u^{n}_{i+1,j} - u^{n}_{i,j}) - q_{i -\frac{1}{2},j}(u^{n}_{i,j} - u^{n}_{i-1,j})}) \\ 
&+  (\frac{\Delta t^2}{\Delta 2y^2} (q_{i,j +\frac{1}{2}}(u^{n}_{i,j+1} - u^{n}_{i,j}) - q_{i ,j-\frac{1}{2}}(u^{n}_{i,j} - u^{n}_{i,j-1}))\\
 &+ \frac{\Delta t^2}{2} f(x_i, y_i, t_i)
\end{align*}


\subsection{Numerical scheme for the boundary}
On the boundary this scheme won't work as one, when adding or subtracting one from one of the indices, risks leaving grid. We discretize the Neumann condition to solve this problem:
\begin{align}
[D_{2x} u]_{0\, j}^n = [D_{2x} u]_{Nx\, j}^n = 0 & \quad \text{for j = 0,1..$N_y$, n = 0,1,..$N_t$} \\
[D_{2y} u]_{i\, 0}^n = [D_{2y} u]_{0,Ny}^n = 0 & \quad \text{for i = 0,1..$N_y$, n = 0,1,..$N_t$}
\end{align}
Which leads to the equations:
\begin{align}
u_{1,j}^n = u_{-1,j}^n ,  u_{N_x-1,j}^n = u_{Nx+1,j}^n  & \quad \text{for j = 0,1..$N_y$, n = 0,1,..$N_t$} \\
u_{i,1}^n = u_{i,-1}^n, u_{i,Ny+1}^n = u_{i,Nx+1}^n & \quad \text{for i = 0,1..$N_y$, n = 0,1,..$N_t$}
\end{align}
A scheme for the borders can now be found by applying each equality to the inner point scheme for its boundary. The corners are found by applying two equalities, one for each of the adjecent border.

\subsection{Approximating q(x,y) outside the grid}
To approximate the q function when evaluated outside the grid we will apply the arithmetic and harmonic mean:
\begin{align}
q_{i+\frac{1}{2},j} &= \frac{q_{i,j} + q_{i+1,j}}{2} \label{armean} \\
q_{i+\frac{1}{2},j} &= 2 (\frac{1}{q_{i,j}} + \frac{1}{q_{i+1,j}})^{-1} \label{harmean}
\end{align}

\section{Implementation}
We implemented the code in pure python. See the wave2D\_du0.py for the full code.

TODO: Show the core of the program in minted enviroment

\section{Numerical experiments}
\end{document}