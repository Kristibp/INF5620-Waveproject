%%
%% Automatically generated ptex2tex (extended LaTeX) file
%% from Doconce source
%% http://code.google.com/p/doconce/
%%




%-------------------------- begin preamble --------------------------
\documentclass[twoside]{article}



\usepackage{relsize,epsfig,makeidx,amsmath,amsfonts}
\usepackage[latin1]{inputenc}
\usepackage{minted} % packages needed for verbatim environments


% Hyperlinks in PDF:
\usepackage[%
    colorlinks=true,
    linkcolor=black,
    %linkcolor=blue,
    citecolor=black,
    filecolor=black,
    %filecolor=blue,
    urlcolor=black,
    pdfmenubar=true,
    pdftoolbar=true,
    urlcolor=black,
    %urlcolor=blue,
    bookmarksdepth=3   % Uncomment (and tweak) for PDF bookmarks with more levels than the TOC
            ]{hyperref}
%\hyperbaseurl{}   % hyperlinks are relative to this root

% Tricks for having figures close to where they are defined:
% 1. define less restrictive rules for where to put figures
\setcounter{topnumber}{2}
\setcounter{bottomnumber}{2}
\setcounter{totalnumber}{4}
\renewcommand{\topfraction}{0.85}
\renewcommand{\bottomfraction}{0.85}
\renewcommand{\textfraction}{0.15}
\renewcommand{\floatpagefraction}{0.7}
% 2. ensure all figures are flushed before next section
\usepackage[section]{placeins}
% 3. enable begin{figure}[H] (often leads to ugly pagebreaks)
%\usepackage{float}\restylefloat{figure}

\newcommand{\inlinecomment}[2]{  ({\bf #1}: \emph{#2})  }
%\newcommand{\inlinecomment}[2]{}  % turn off inline comments

% insert custom LaTeX commands...

\makeindex

\begin{document}
%-------------------------- end preamble --------------------------





% ----------------- title -------------------------

\begin{center}
{\LARGE\bf Effects of Hill Shapes \\ [1.5mm] on Tsunami Simulations}
\end{center}




% ----------------- author(s) -------------------------

\begin{center}
{\bf Kristian Pedersen and Gustav Baardsen} \\ [0mm]
%{\bf Hans Petter Langtangen${}^{1, 2}$ (\texttt{hpl@simula.no})} \\ [0mm]
\end{center}

%\begin{center}
% List of all institutions:
%\centerline{{\small ${}^1$Center for Biomedical Computing, Simula Research Laboratory}}
%\centerline{{\small ${}^2$Department of Informatics, University of Oslo.}}
%\end{center}
% ----------------- end author(s) -------------------------



% ----------------- date -------------------------


\begin{center}
\today
\end{center}

\vspace{1cm}



\begin{abstract}

In this project we study how the shape of hills at the sea bottom
affects numerical and physical properties of simulations of
 earthquake-generated tsunamis. The tsunamis are modelled using 
two-dimensional wave equations, and a finite difference scheme is 
used to solve the partial differential equations.


\end{abstract}

\tableofcontents



% Section with multi-line equation.
\section{Introduction}

\label{Introduction}
In this text we will address a two-dimensional, standard linear wave equation, with damping and reflecting boundaries. We will develop a scheme for solving it and apply it to the problem of a tsunami over an uneven seabed. Especial focus will be put on the question of which kinds of seabeds causes numerical instability. 

\section{Mathematical model}

\label{Model}
For $(x, y)$ in the domain $D = [0,L_y] \times [0,L_x]$ and $t \in [0,T]$, we will study the partial differential equation
\begin{equation}
u_{tt} + b u_{t} = (q(x, y)u_{x})_{x} + (q(x, y)u_{y})_{y} + f(x, y, t), 
\end{equation}
where $(x, y) \in D$ and $t \in [0, T]$, with the boundary and initial conditions
\begin{align}
\frac{\partial u}{\partial n} &= 0 & (x, y) \in \partial D,  t \in [0, T], \nonumber \\
u(x, y, 0) &= I(x, y) &x,y \in \partial D, \nonumber \\
u_{t}(x, y, 0) &= V(x, y) &x,y \in \partial D.
\end{align} 
Here $\partial D$ denotes the boundary of the domain $D$ and $\partial /\partial n$ stands for differentiation in the normal direction out of the boundary.

\section{Numerical scheme}

\label{Scheme}
\subsection{Numerical scheme for the inner points}
In operator notation, we want to use the following scheme for inner points:
\begin{align}
[D_{t} D_{t} u + b D_{2t} u = D_{x}(q D_{x} u) + D_{y} (q D_{y} u) +f]_{i\, j}^{n} \label{PDE}
\end{align}
Calculating each summand we find:

\begin{align}
[D_{t} D_{t} u]_{i\, j}^{n} &= \frac{u^{n+1}_{i, j} - 2u^{n}_{i, j} + u^{n-1}_{i, j}}{\Delta t^{2}} \label{Dttu}, \\
[bD_{2}t u]_{i\, j}^n &= b\frac{u^{n+1}_{i,j} - u^{n-1}_{i,j}}{2\Delta t} \label{bD2tu}, \\
[D_x q D_x u]_{i\, j}^n &= \frac{q_{i +\frac{1}{2},j}(u^{n}_{i+1,j} - u^{n}_{i,j}) - q_{i -\frac{1}{2},j}(u^{n}_{i,j} - u^{n}_{i-1,j})}{\Delta x^2} \label{DxqDxu}, \\
[D_y q D_y u]_{i\, j}^n &= \frac{q_{i,j +\frac{1}{2}}(u^{n}_{i,j+1} - u^{n}_{i,j}) - q_{i ,j-\frac{1}{2}}(u^{n}_{i,j} - u^{n}_{i,j-1})}{\Delta x^2} \label{DyqDyu}, \\
[f]_{i\, j}^n &= f(x_i, y_{j}, t_{n}). \label{f}
\end{align}
Inserting these expressions into Eq. (\ref{PDE}) and solving for $u^{n+1}_{i,j}$, we get the following scheme for inner points:
\begin{align}
u^{n+1}_{i,j} &= \left\{ 2u^{n}_{i,j} + u^{n-1}_{i,j} \right. \nonumber \\
& \left. + \frac{\Delta t^2}{\Delta x^2}\left(q_{i +\frac{1}{2},j}(u^{n}_{i+1,j} - u^{n}_{i,j}) - q_{i -\frac{1}{2},j}(u^{n}_{i,j} - u^{n}_{i-1,j})\right) \right. \nonumber \\ 
&\left. +  \frac{\Delta t^2}{\Delta y^2} \left(q_{i,j +\frac{1}{2}}(u^{n}_{i,j+1} - u^{n}_{i,j}) - q_{i ,j-\frac{1}{2}}(u^{n}_{i,j} - u^{n}_{i,j-1})\right) \right. \nonumber \\
&\left. + \Delta t^{2} f(x_i, y_j, t_n) \right\}/\left( 1 + \frac{1}{2}b\Delta t \right)
\end{align}

\subsection{Numerical scheme for the first time step}
For the first time step $n = 0$, $u^{n-1}_{i,j}$ is not a part of the grid. To circumvent this problem, we have to apply the intial condition $u_t(x,y,0) = V(x,y) \text{ for } x,y \in \partial D$. We discretize it and get:
$$[D_{2t} u = V]_{i\,j}^0 \implies \frac{u_{i\,j}^1 - u_{i\,j}^{-1}}{2\Delta t} = V(x_i,y_j) \implies u_{i\,j}^{-1} = u_{i\,j}^{1} - 2\Delta t V(x_i,y_j).$$
Inserting this into equation (\ref{Dttu}) and (\ref{bD2tu}), we get:
\begin{align}
[D_t D_t u]_{i\, j}^n &= 2\left(\frac{u^{n+1}_{i,j} - u^{n}_{i,j}}{\Delta t^2}  - \frac{V(x_i,y_i)}{\Delta t}\right), \\
[bD_{2t} u]_{i\, j}^n &= b\frac{V(x_i,y_i)}{\Delta t}. \label{bD2tu2}
\end{align}
Inserting these expressions into Eq. (\ref{PDE}) and solving for $u^{n+1}_{i,j}$, we get the following scheme for inner points in the first time step $n = 0$:
\begin{align}
u^{n+1}_{i,j} &= u^{n}_{i,j} + \left( 1 - \frac{1}{2}b\Delta t\right)\Delta t V_{i,j} \nonumber \\
&+ \frac{\Delta t^2}{2\Delta x^2} \left( q_{i +\frac{1}{2},j}(u^{n}_{i+1,j} - u^{n}_{i,j}) - q_{i -\frac{1}{2},j}(u^{n}_{i,j} - u^{n}_{i-1,j})\right) \nonumber \\ 
&+  \frac{\Delta t^2}{2\Delta y^2} \left( q_{i,j +\frac{1}{2}}(u^{n}_{i,j+1} - u^{n}_{i,j}) - q_{i ,j-\frac{1}{2}}(u^{n}_{i,j} - u^{n}_{i,j-1})\right) \nonumber \\
 &+ \frac{\Delta t^2}{2} f(x_{i}, y_{j}, t_{n}).
\end{align}


\subsection{Numerical scheme for the boundary}
On the boundary, this scheme needs points outside the defined space domain. For example, we need values for $u_{-1, j}^{n}$, $u_{i, N_{y}+1}^{n}$, and $u_{0, -1}^{n}$.
Values for these points can be obtained from the discretized versions of the Neumann condition. The Neumann conditions are in discretized form
\begin{align}
[D_{2x} u]_{0\, j}^n = [D_{2x} u]_{N_{x}\, j}^n = 0 & \quad \text{ for } j = 0, 1,\ldots , N_{y}, \quad n = 0, 1, \dots , N_{t}, \\
[D_{2y} u]_{i\, 0}^n = [D_{2y} u]_{0,N_{y}}^n = 0 & \quad \text{ for } i = 0, 1, \dots , N_y, \quad n = 0, 1, \dots , N_{t},
\end{align}
which leads to the equations
\begin{align}
u_{1,j}^n = u_{-1,j}^n ,  \quad u_{N_x-1,j}^n = u_{Nx+1,j}^n  & \quad \text{ for } j = 0, 1, \ldots , N_{y}, \quad n = 0, 1, \ldots , N_{t} \\
u_{i,1}^n = u_{i,-1}^n, \quad u_{i,Ny+1}^n = u_{i,Nx+1}^n & \quad \text{ for } i = 0, 1, \ldots , N_{y},\quad n = 0, 1, \ldots , N_{t}.
\end{align}
A scheme for the borders can now be found by applying each equality to the inner point scheme for its boundary. The corners are found by applying two equalities, one for each of the adjecent border.

\subsection{Approximating q(x,y) outside the grid}

In our implementation, we assume that values for the function $q(x, y)$ are given only at the grid points $(x_{i}, y_{j})$. In the finite difference algorithm, we use mean values to evaluate the function at other points. To approximate the $q$ function when evaluated outside the grid, we will apply the arithmetic and harmonic mean, defined respectively as
\begin{align}
  q_{i+\frac{1}{2},j} &= \frac{q_{i,j} + q_{i+1,j}}{2} \label{armean} \\
\end{align}
and
\begin{align}
  q_{i+\frac{1}{2},j} &= 2 \left(\frac{1}{q_{i,j}} + \frac{1}{q_{i+1,j}}\right)^{-1}. \label{harmean}
\end{align}
When not states explicitly otherwise, we have used the arithmetic mean as the default option.


\section{Implementation}

We implemented the code in pure python. See the wave2D\_du0.py for the full code.

TODO: Show the core of the program in minted enviroment


%% The numerical method is implemented in a Python function:

%% \begin{minted}[fontsize=\fontsize{9pt}{9pt},linenos=false,mathescape,baselinestretch=1.0,fontfamily=tt,xleftmargin=7mm]{python}
%% def theta_rule(I, a, T, dt, theta):
%%     """Solve u'=-a*u, u(0)=I, for t in (0,T] with steps of dt."""
%%     N = int(round(T/float(dt)))  # no of intervals
%%     u = zeros(N+1)
%%     t = linspace(0, T, N+1)

%%     u[0] = I
%%     for n in range(0, N):
%%         u[n+1] = (1 - (1-theta)*a*dt)/(1 + theta*dt*a)*u[n]
%%     return u, t
%% \end{minted}

% Section with figures.


\section{Numerical experiments}

%\index{numerical experiments}

\subsection{Verification}

The implementation of the finite difference algorithm was verified with the following tests:  

\begin{itemize}
\item In the first test the initial conditions were set to 
  \begin{align}
    u(x, y, t=0) & \equiv I(x, y) = u_{0}, \nonumber \\
    u_{t}(x, y, t=0) & \equiv V(x, y) = 0,
  \end{align}

  where $u_{0}$ is a constant, and we used the restriction $f(x, y, t) = 0$. The resulting PDE gives the constant solution $u(x, y, t) = u_{0}$.
\item In the second test case, the 2-dimensional code was tested with the simple one-dimensional plug function 
  \begin{displaymath}
  I(x, y) = \left\{ \begin{array}{ll}
    0 & \text{ if } |x - L/2| > a, \\
    1 & \text{ else, }
  \end{array} \right.
\end{displaymath}

as initial condition. In the case with $V(x, y) = 0$, $f(x, y, t) = 0$, and
$q(x, y) = c^{2}$, where $c$ is a constant, the solution $u(x, y, t)$ gives
exactly two moving squares, as long as the Carnot number $C \equiv c\Delta t/ \Delta x$ is 1. This test was repeated by interchanging $x$ and $y$ in the initial condition.

\item As another one-dimensional test, we used the one-dimensional initial conditions
  \begin{align}
    I(x, y) & = \exp\left(a(x - L/2)^{2}\right), \nonumber \\
    V(x, y) & = 0,  
  \end{align}
where $a$ is a constant, togehter with the restrictions $b = 0$ and $f(x, y, t) = 0$. These conditions give the solution
\begin{align}
  u(x, y, t) &= \frac{1}{2}\exp\left(-a(x-ct-L_{x}/2)^{2}\right) \nonumber \\
  & = + \frac{1}{2}\exp\left(-a(x + ct - L_{x}/2\right)^{2}),  
\end{align} 
at the limit when the boundaries are infinitely far away. This solution does not fulfill the boundary condition $\partial u/\partial n = 0$, but it was a useful test for the first few steps of the simulation.

\item As suggested in the project description, we also used the manufactured solution 
  \begin{equation}
    u(x, y, t) = \exp\left( -bt\right)\cos\left( \omega t\right)\cos\left( \frac{m_{x}x\pi }{L_{x}}\right)\cos\left( \frac{m_{y}y\pi }{L_{y}}\right),
  \end{equation}
where $b$ is the damping parameter in the studied PDE, $\omega $ is a real constant, and $m_{x}$ and $m_{y}$ are integers. The solution $u(x, y, t)$ is a standing wave with the desired boundary condition $\partial u/\partial n = 0$. This manufactured solution was tested both with constant $q$ and with the choise $q(x, y) = \exp(-x-y)$. For both cases we had to determine functions $f(x, y, t)$ such that the given manufactured solution fulfills the two-dimensional wave equation.
According to the project formulation, the error $\varepsilon $ of the solution $u(x, y, t)$ should converge as  
\begin{equation}
  \varepsilon = Dh^{2},
\end{equation}
where
\begin{equation}
  D = D_{t}F_{t}^{2}+D_{x}F_{x}^{2}+D_{y}F_{y}^{2},
\end{equation}
$\Delta t = F_{t}h $, $\Delta x = F_{x}h$, $\Delta y = F_{y}h$, and $D_{i}$ and $F_{i}$, $i \in \{ t, x, y \}$, are constants.

In our implementation, the convergence rate $r$ was estimated by assuming the relation between rates of subsequent errors $\varepsilon_{k}$ with decreasing parameter $h_{k}$ and rates of the convergence parameter $h_{k}$ being of the form
\begin{equation}
  \frac{\varepsilon_{h_{k-1}}}{\varepsilon_{h_{k}}} = \left(\frac{h_{k-1}}{h_{k}}\right)^{r}.
\end{equation}
Here the considered error was chosen to be 
\begin{equation}
  \varepsilon_{h_{k}} = \max_{i, j} \left|u_{e}(x_{i}, y_{j}, t_{N})-u_{h_{k} i,j}^{N}\right|,
\end{equation}
where $u_{e}(x, y, t)$ is the exact solution and $u_{h_{k} i,j}^{N}$ is the numerical solution with convergence parameter $h_{k}$. Examples of calculated convergence rates are given in Table \ref{tab:conv}.

\end{itemize}

  \begin{table} \label{tab:conv}
    \begin{center}
      \caption{Convergence rates $r_{k}$ for different convergence parameters $h_{k}$ and two different choises of the function $q(x, y)$.}
      \begin{tabular}{ccc}
        \hline\hline
        $h_{k}$ & $q = $ const. & $q(x, y) = \exp(-x-y)$ \\
        \hline
        0.01 & 1.959 & 1.984 \\
        0.001 & 1.996 & 1.998 \\
        \hline\hline

      \end{tabular}
    \end{center}
  \end{table}

  \subsection{One-dimensional simulations}
  
  We did some simulations with one-dimensional systems to get a better view of 
  the behaviour of the solutions. With the one-dimensional systems, it was also possible to use finer mesh grids. In the directories \verb+movie_1dwave_box_Ba250+ and \verb+movie_1dwave_gaussian_Bs250+ we have examples of simulations of one-dimensional systems with a box and a Gaussian shaped hill, respectively. With the same hight of the hill, the simulation with a Gaussian perturbation gave a smoother function $u(x, y, t)$ than what was the case with the box-shaped perturbation.  

\begin{figure} 
  \centering
  \includegraphics[scale=0.4]{gustavs_codes/movie_1dwave_box_Ba250/figure.pdf}
  \caption{A snapshot of a one-dimensional simulation with a box-shaped hill at the bottom of the sea. The solution is clearly not smooth, and has probably numerical noise.} %\label{fig:}
\end{figure}

\begin{figure} 
  \centering
  \includegraphics[scale=0.4]{gustavs_codes/movie_1dwave_gaussian_Bs250/figure.pdf}
  \caption{A picture from a one-dimensional simulation with a Gaussian-shaped hill at the seabed. This solution is smoohter than the result of the simulation with a box-sahped hill, but is partially quite sharp.} %\label{fig:}
\end{figure}

\section{Conclusions}

%\begin{figure}[!ht]
%  \centerline{\includegraphics[width=0.9\linewidth]{error.png}}
%  \caption{
%  Error versus time step. \label{fig:E}
%  }
%\end{figure}
%\clearpage % flush figures fig:E

\printindex

\end{document}
